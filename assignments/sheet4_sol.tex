%%%%%%%%%%%%%%%%%%%%%%%%%%%%%%%%%%%%%%
% Numerical Linear Algebra class 2022 
% Solutions to Sheet 4
%%%%%%%%%%%%%%%%%%%%%%%%%%%%%%%%%%%%%%

\begin{SolutionSheet}[\ref{sheet4}]
\begin{onehalfspace}
  
  \begin{Solution}
    \Claim The QR-iteration preserves the tridiagonal structure of a symmetric tridiagonal matrix. \\
    \Proof We know for the QR-iteration $A_{k+1} = R_kQ_k$ and $A_k = Q_kR_k$ \\
      $\implies A_{k+1} = Q_k^*A_kQ_k \\
      \implies A_{k+1} = U_k^*A_0 U_k$ with $U_k = Q_1 \dot ... \dot Q_k \\
      \implies$ with $A_0$ is $A_{k+1}$ also hermitian: \begin{equation*}
        A_{k+1}^* = (U_k^*A_0 U_k)^T = U_k^T A_0^T (U_k^*)^T = 
        \overline{U_k^*} A_0 \overline{U_k} = \overline{A_{k+1}}
      \end{equation*}
      With Lemma B.1.2 we know: \begin{equation*}
        span{a_1,...,a_i} = span{q_1,...,q_i} \quad i=1,...,n
      \end{equation*}
      and $a_k = \sum_{i=1}^n r_{ik}q_i \\
      \\
      A_k = \begin{pmatrix}
        \ast & \ast &&& \\
        \ast & \ddots & \ddots &0& \\
        &\ddots&\ddots&\ddots&\\
        & 0 &\ddots & \ddots &\ast \\
        & & &\ast & \ast 
      \end{pmatrix} = \underbrace{\begin{pmatrix}
        \ast & & && \\
        \ast & \ddots && \ast& \\
        &\ddots& \ddots&&\\
        &0 & \ddots& \ddots\\
        & && \ast & \ast
      \end{pmatrix}}_{Q_k} \underbrace{\begin{pmatrix}
        \ast & \ast &\ast && 0 \\
        &\ddots&\ddots&\ddots& \\
        && \ddots & \ddots & \ast \\
        & 0 && \ddots &\ast \\
        & &&  & \ast
      \end{pmatrix}}_{R_k} \\
      \\
      \underbrace{\begin{pmatrix}
        \ast & \ast &\ast && 0 \\
        &\ddots&\ddots&\ddots& \\
        && \ddots & \ddots & \ast \\
        & 0 && \ddots &\ast \\
        & &&  & \ast
      \end{pmatrix}}_{R_k} \underbrace{\begin{pmatrix}
        \ast & & && \\
        \ast & \ddots && \ast& \\
        &\ddots& \ddots&&\\
        &0 & \ddots& \ddots\\
        & && \ast & \ast
      \end{pmatrix}}_{Q_k} = \begin{pmatrix}
        \ast & & && \\
        \ast & \ddots && \ast& \\
        &\ddots& \ddots&&\\
        &0 & \ddots& \ddots\\
        & && \ast & \ast
      \end{pmatrix} = A_{k+1}\\
      \\
      \stackrel{A_{k+1} \text{hermitian}}{\implies} A_{k+1}$ is tridiagonal
  \end{Solution}

  \begin{Solution}
    \Claim For a symmeric matrix $A$, in the proof of Theorem  1.4.16, $H$ would be tridiagonal \\
    \Proof Any matrix $A\in \C ^{n\x n}$ can be transformed into a hessenberg matrix
    using similarity transformations st $H=Q^TAQ$ (with householder reflections for example). When A is symmetric, $H$ is also 
    symmetric and therefore tridiagonal.  
  \end{Solution}

  \begin{Solution}
    Use Algorithm B.1.10 to compute the QR-factorization
  \end{Solution}

  \begin{Solution}
    The QR method will not converge for $A=$ \smatrix{0}{1}{1}{0}, since $A$ is orthogonal \\
    $\implies Q=A, \quad R=\I \\
    \stackrel{inductive}{\implies} A^{(k+1)} = R^{(k)}Q^{(k)} = \I A^{(k)} = A^{(k)}$
  \end{Solution}

  \begin{Solution}[Programming]
    The eigenvalues of $\mata_n$ are
    \begin{gather*}
      \lambda_{n,j} = 4\sin^2\left(\frac{j\pi}{2n+2}\right)
    \end{gather*}
  https://math.stackexchange.com/questions/3875168/eigenvalues-of-a-tridiagonal-matrix-with-1-2-1-as-entries

  https://math.stackexchange.com/questions/177957/eigenvalues-of-tridiagonal-symmetric-matrix-with-diagonal-entries-2-and-subdiago?rq=1
  \end{Solution}

\end{onehalfspace}
\end{SolutionSheet}


%%% Local Variables: 
%%% mode: latex
%%% TeX-master: "main"
%%% End: 
