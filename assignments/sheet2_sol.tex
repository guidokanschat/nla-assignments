%%%%%%%%%%%%%%%%%%%%%%%%%%%%%%%%%%%%%%
% Numerical Linear Algebra class 2022 
% Solutions to Sheet 2
%%%%%%%%%%%%%%%%%%%%%%%%%%%%%%%%%%%%%%

\begin{SolutionSheet}[\ref{sheet2}]

  \begin{Solution}
    \textit{Claim:} $A$ normal $\ \iff A = Q^{-1}BQ \ $ with $Q$ unitary, $B$ normal
    \textit{Proof:} $"\implies ": \ A$ is normal, $A \sim A$ \\
    $"\Leftarrow ":$ Let $A$ be a unitarily similar to $B$, $B$ normal \\
    $ \implies BB^* = B^*B, \quad\exists Q \in U(n): A = Q^{-1}BQ$ \\
    $ \implies$ \begin{align*}
      AA^* &= Q^{-1}BQ(Q^{-1}BQ)^* \\
      &= Q^{-1}BQ Q^* B^* (Q^{-1})^* \\
      &= Q^{-1} BB^*Q \\
      &= Q^{-1} B^*BQ \\
      &= Q^{-1} B^* QQ^{-1}BQ \\
      &= (Q^* B^* Q)(Q^{-1}BQ) \\
      &= (Q^* B Q)^* A \\
      &= A^* A
    \end{align*}
    $ \implies A$ normal
  \end{Solution}

  \begin{Solution}
    \textit{Claim:} $A$ normal, $Av = \lambda v \ 
      \implies \ v^* A = \lambda v^*$ \\
    \textit{Proof:} Let $A$ be a normal matrix. \\
    $ $ note: \begin{equation*}
      \begin{split}
        \norm{Ax} = 0 \ &\iff \ \norm{Ax}^2 = 0 \quad 
          \iff \ \langle Ax,Ax \rangle = 0 \\
          &\iff \ \langle x, A^*Ax \rangle = 0 \ 
          \iff \ \langle x, AA^*x \rangle = 0 \\
          &\iff \  \langle A^*x,A^*x\rangle = 0 \: 
          \iff \ \norm{A^*x}^2 = 0 \\
          &\iff \ \norm{A^*x} = 0
      \end{split}
    \end{equation*}
    $\implies$ \begin{equation*}
      (A - \lambda \mathbb{I})^* 
        = (\overline{\rm A} - \overline{\rm \lambda \mathbb{I}})^T 
        = (A^* - \overline{\lambda} \mathbb{I}) 
    \end{equation*}
    $ \implies$ \begin{align*} 
      (A - \lambda \mathbb{I})(A - \lambda \mathbb{I})^* &= (A - \lambda \mathbb{I})(A^* - \overline{\lambda} \mathbb{I}) \\
      &= AA^* - \lambda A^*\mathbb{I} - \overline{\lambda} A \mathbb{I} - \lambda \overline{\lambda} \mathbb{I} \\
      &= A^*A - \overline{\lambda} A \mathbb{I} - \lambda A^*\mathbb{I} - \lambda \overline{\lambda} \mathbb{I} \\
      &= (A^* - \overline{\lambda} \mathbb{I})(A - \lambda \mathbb{I}) \\
      &= (A - \lambda \mathbb{I})^* (A - \lambda \mathbb{I})
    \end{align*}
    $ \implies A - \lambda \mathbb{I}$ is normal\\
    \\
    Let $v$ be a (right) eigenvector to $\lambda$ \\
    $\begin{array}{ll}      
      \implies &(A- \lambda \mathbb{I})v = 0 \\
      \implies &(A- \lambda \mathbb{I})^*v = 0 \\
      \implies &A^*v = \overline{\lambda} \mathbb{I}v \\
      \implies &v^T(A^*)^T = \overline{\lambda} v^T \\
      \implies &v^T \overline{A} = \overline{\lambda} v^T \\
      \implies &v*A = \lambda v^*
    \end{array}$

  \end{Solution}

  \begin{Solution} 
    $M = \big(\begin{smallmatrix}
      \eta & 1\\
      \eta & \eta
    \end{smallmatrix}\big)$ with $|\eta| << 1 \\
    \begin{array}{ll}
      \implies & \lambda_{1,2}(M) = \eta \pm \sqrt{\eta}, \quad v_1 = \big(\begin{smallmatrix} \sqrt{\pm\eta} \\
        1 \end{smallmatrix} \big) \\
      \implies & v_1 \text{almost parralel to} v_2 
    \end{array}$
    
    Take \begin{equation*} \widetilde{M} := M + \Delta M \end{equation*}
    so $\widetilde{M}$ is a small change to $M$. \\
    \\
    $\begin{array}{lll} 
      e.g. &\Delta M = -(\eta) \bigl( \begin{smallmatrix} 1 & 0 \\ 1&1 \end{smallmatrix} \bigr) \\
      &\implies \widetilde{M} = \bigl( \begin{smallmatrix} 0 & 1 \\ 0&0 \end{smallmatrix} \bigr) \\
      &\implies \lambda_{1,2}(\widetilde{M}) = 0, \
      &v_{1,2}(\widetilde{M}) = \bigl( \begin{smallmatrix} 1 \\ 0  \end{smallmatrix} \bigr) \implies \text{not conituous}
    \end{array}$
  \end{Solution}

  \begin{Solution} \phantom{A}\\
    \begin{equation*}
      A =\begin{pmatrix} \cos\phi & -\sin\phi\\
        \sin\phi &  \cos\phi  \end{pmatrix}^T
      \begin{pmatrix}  1 & \\
          & c \end{pmatrix}
      \begin{pmatrix} \cos\phi & -\sin\phi\\
        \sin\phi &  \cos\phi \end{pmatrix} =: R^T A R
      \end{equation*}
    \begin{enumerate}
      \item $R$ orthogonal \\
        $\stackrel{\text{Lemma 1.1.13}}{\implies} A$ diagonalizable, $R$ orthogonal basis of eigenvectors \\
        $\implies \lambda_1 = 1, \ \lambda_2 = c, \quad
        v_1 = \bigl( \begin{smallmatrix} \cos\phi \\ \sin\phi  \end{smallmatrix} \bigr)
        v_2 = \bigl( \begin{smallmatrix} -\sin\phi \\ \cos\phi  \end{smallmatrix} \bigr)$
      \item (Programming)
      \item $c=1 \ \implies \ A = \mathbb{I} \ \implies \ x^{(i)} = x^{(0)} \ \forall i$. \\
        Since $A^n$ converges (see (4)), $A^nx$ converges, too.
      \item \begin{align*}
        lim_{n \to \infty} A^n &\stackrel{\text{R orthogonal}}{=} 
        R^T lim_{n \to \infty}   \bigl( \begin{smallmatrix} 1 & \\ & c \end{smallmatrix} \bigr) R \\
         &= R^T  \bigl( \begin{smallmatrix} 1 & \\ & 0  \end{smallmatrix} \bigr) R \\
         &= \bigl( \begin{smallmatrix} (\cos\phi)^2 & -\sin\phi\cos\phi \\ -\sin\phi\cos\phi & (\sin\phi)^2  \end{smallmatrix} \bigr)
      \end{align*}
    \end{enumerate}
  \end{Solution}

\end{SolutionSheet}


%%% Local Variables: 
%%% mode: latex
%%% TeX-master: "main"
%%% End: 
